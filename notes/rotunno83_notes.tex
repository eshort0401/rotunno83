\documentclass[12pt]{article}
\usepackage{hyperref}
\bibliographystyle{agsm}
\usepackage{har2nat}
\usepackage{amsmath,amssymb,amsthm,amscd,verbatim,graphicx,color}
\hypersetup{colorlinks=true, urlcolor=blue, citecolor=black, linkcolor=black}
\usepackage[x11names, rgb]{xcolor}
\usepackage[utf8]{inputenc}
\usepackage[a4paper, margin=1.5cm]{geometry}
\usepackage[font={small, stretch=1.3}, labelfont=bf]{caption}
\setlength{\parskip}{0em}
\renewcommand{\floatpagefraction}{.7}
\linespread{1.3}

\DeclareMathOperator{\sgn}{sgn}

\title{Notes concerning ``On the Linear Theory of the Land and Sea Breeze"  \citep{rotunno83}}
\author{Ewan Short}
\date{\today}

\begin{document}

\maketitle

\section{Paper Misprints}
\begin{enumerate}
\item
Equation (37) should be 
\begin{equation*}
 -\beta \textrm{Re}\left\{ \mathcal{F}^{-1}\left[ \frac{1}{k}e^{-i\sgn(k) k \zeta} \int_0^\zeta \mathcal{F}\left(\frac{\partial \tilde{Q}}{\partial \xi}\right) \sin k\zeta' d\zeta' + \frac{1}{k}\sin k \zeta \int_\zeta^\infty \mathcal{F}\left(\frac{\partial \tilde{Q}}{\partial \xi}\right) e^{-i\sgn(k)k\zeta'} d\zeta' \right] \right\}.
\end{equation*}
\item
$\tilde{Q}$ should have time dependance $e^{-i\left(\tau-\frac{\pi}{2}\right)}=ie^{-i\tau}$ not simply $\sin\tau$. 
\item
The expression $\tilde{b} = b h^{-1} \omega^{-3}$ in equation (13) should actually be $\tilde{b} = b h^{-1} \omega^{-2}$ to make the units come out right - assuming $b$ has units m s$^{-2}$.
\end{enumerate}

\section{Mid-Latitude Case}
We begin with equation (14)
\begin{equation*}
\frac{\partial^2 \tilde{\psi}}{\partial \xi^2} + \frac{\partial^2 \tilde{\psi}}{\partial \zeta^2} = -\beta \frac{\partial \tilde{Q} }{\partial \xi}.
\end{equation*}
Consider the Green's function defined by
\begin{align*}
\frac{\partial^2 g}{\partial \xi^2} + \frac{\partial^2 g}{\partial \zeta^2} = \delta(\xi-\xi')\delta(\zeta-\zeta')
\end{align*}
where $\xi,\zeta$ can vary over all of $\mathbb{R}$. Integrating over a circle $C$ centred at $\xi',\zeta'$ with radius $a$, where $r$ is the distance from $(\xi',\zeta')$ 
\begin{align*}
& \Rightarrow \iint_C \nabla \cdot \nabla g dV = 1 \\
& \Rightarrow \int_{\partial C} \frac{\partial g}{\partial r} = 1.
\end{align*}
If there are no boundaries, then the response of $g$ to singular forcing at $(\xi',\zeta')$ is unnaffected by rotation, and is therefore constant on $\partial C$. Thus 
\begin{align*}
& \Rightarrow 2\pi a \left.\frac{\partial g}{\partial r}\right|_a = 1.
\end{align*}
Because $a$ was arbitrary we have therefore have
\begin{align*}
& \Rightarrow \frac{\partial g}{\partial r} = \frac{1}{2\pi r} \\
& \Rightarrow g = \frac{\ln\left(r \right)}{2\pi} + c = \frac{1}{4\pi}\ln\left(\left(\xi-\xi'\right)^2+\left(\zeta-\zeta'\right)^2\right) + c.
\end{align*}
Can assume without loss of generality that $c=0$, because $c\ne 0$ simply adds a constant to $\tilde{\psi}$, and we only care about the $\xi$ and $\zeta$ derivatives of $\tilde{\psi}$. 

\section{Tropical Case}

\subsection{Deriving Equation (37)}
Equation (36) gives
\begin{align*}
\frac{\partial ^2 \tilde{\psi}}{\partial \xi^2} - \frac{\partial ^2 \tilde{\psi}}{\partial \zeta^2} = -\beta \frac{\partial^2 \tilde{Q}}{\partial \xi^2}.
\end{align*}
Take Fourier transform to get
\begin{align*}
& \phantom{=} -k^2 \widehat{\tilde{\psi}} - \frac{\partial ^2 \widehat{\tilde{\psi}}}{\partial \zeta^2} = -\beta \frac{\partial^2 \widehat{\tilde{Q}}}{\partial \xi^2} \\
&= k^2 \widehat{\tilde{\psi}} + \frac{\partial ^2 \widehat{\tilde{\psi}}}{\partial \zeta^2} = \beta \frac{\partial^2 \widehat{\tilde{Q}}}{\partial \xi^2}.
\end{align*}
Boundary condition becomes $\left(ik\right)\widehat{\tilde{\psi}}(k,0)=0$, and so $\widehat{\tilde{\psi}}(k,0)=0$ (using the Fourier transform rule for derivatives.) Solve for Green's Function
\begin{align*}
&\phantom{=} k^2 G + G_{\zeta\zeta} = \delta\left(\zeta-\zeta'\right).
\end{align*}
General solution setting RHS to zero is $G=B_1 e^{i k\zeta} + B_2 e^{-ik\zeta}$. For $\zeta < \zeta'$ the boundary condition gives
\begin{align*}
0=B_1+B_2 & \Rightarrow B_2 = -B_1 \\
& \Rightarrow G = B_1 2i \sin(k\zeta) = A \sin(k\zeta).
\end{align*}
Now consider $\zeta > \zeta'$. For $k>0$ we have
\begin{align*}
G e^{-i\left(\tau-\frac{\pi}{2}\right)} &= B_1 e^{i k\zeta-i\tau+i\frac{\pi}{2}} + B_2 e^{-ik\zeta-i\tau+i\frac{\pi}{2}} \\
&= B_1 e^{i \left(k\zeta-\tau+\frac{\pi}{2}\right)} + B_2 e^{-i\left(k\zeta+\tau-\frac{\pi}{2}\right)}.
\end{align*}
Recall for gravity waves, energy propagates with the group velocity in the \textit{opposite} direction to the phase velocity. As $\zeta>\zeta'$ we require positive group velocity, and therefore negative phase velocity. Thus $B_1=0$ and $G=B_2e^{-ik\zeta}$. Similarly for $k<0$ we have $G=B_1e^{ik\zeta}$. Thus $G=Be^{-i\sgn(k)k\zeta}$.

Now, continuity requires that 
\begin{align*}
\lim_{\zeta\to\zeta'^+} G = \lim_{\zeta\to\zeta'^-} G \\
\lim_{\zeta\to\zeta'^+} G_\zeta - \lim_{\zeta\to\zeta'^-} G_\zeta = 1. 
\end{align*}
Thus 
\begin{align*}
&\phantom{\Rightarrow} Be^{-i\sgn(k)k\zeta'} = A\sin(k\zeta') \\
&\phantom{\Rightarrow} -i\sgn(k)k Be^{-i\sgn(k)k\zeta'} - Ak\cos(k\zeta') = 1 \\
&\Rightarrow i\sgn(k)k A\sin(k\zeta') + Ak\cos(k\zeta') = -1.
\end{align*}
Now, $k>0$
\begin{align*}
&\Rightarrow ik A\sin(k\zeta') + Ak\cos(k\zeta') = A k e^{ik\zeta'} = -1 \\
& \Rightarrow A = -\frac{1}{k} e^{-ik\zeta'}.
\end{align*}
Also, $k<0$
\begin{align*}
&\Rightarrow -ik A\sin(k\zeta') + Ak\cos(k\zeta') = A k e^{-ik\zeta'} = -1 \\
& \Rightarrow A = -\frac{1}{k} e^{ik\zeta'}.
\end{align*}
Thus in both cases $A = -\frac{1}{k} e^{-\sgn(k) k \zeta'}$. Thus $B=-\sin(k\zeta')$, and so
\begin{equation*}
G=
\begin{cases} -\frac{1}{k} e^{-i\sgn(k)k\zeta'} \sin(k\zeta), & \zeta<\zeta', \\
-\frac{1}{k}\sin(k\zeta')e^{-i\sgn(k)k\zeta}, & \zeta>\zeta'. 
\end{cases}
\end{equation*}

\subsection{Working Through Fourier Transform}
We are attempting to derive the equation
\begin{equation}
\tilde{\psi} (\xi,\zeta,\tau) = -\beta \tilde{A} \int_0^\infty \frac{\cos k\xi e^{-\xi_0 k }}{1+k^2} \left(\sin(k\zeta + \tau) - e^{-\zeta}\sin\tau \right) dk. \label{Eq:target}
\end{equation}
Note
\begin{align}
&\phantom{\Rightarrow} \tilde{Q} = \beta \tilde{A} \left(\frac{\pi}{2} +\tan^{-1} \frac{\xi}{\xi_0} \right)e^{-\zeta} i e^{-i\tau} \\
&\Rightarrow \frac{\partial \tilde{Q}}{\partial \xi} = \frac{1}{\xi^2 + \xi_0^2} \beta \tilde{A} \xi_0 e^{-\zeta} i e^{-i\tau} \\
&\Rightarrow \mathcal{F}\left(\frac{\partial \tilde{Q}}{\partial \xi}\right) = \mathcal{F}\left[ \frac{1}{\xi^2 + \xi_0^2} \right] \tilde{A} \xi_0 e^{-\zeta} i e^{-i\tau} \\
&\Rightarrow \mathcal{F}\left(\frac{\partial \tilde{Q}}{\partial \xi}\right) =  \frac{\pi}{\xi_0}e^{-\xi_0 |k|} \tilde{A} \xi_0 e^{-\zeta} e^{-i\tau} = \pi e^{-\xi_0 |k|} \tilde{A} e^{-\zeta} i e^{-i\tau}
\end{align}
using the non-unitary, angular frequency form of the Fourier transform. This Fourier transform can be derived by considering the Fourier transform of $\frac{\pi}{\xi_0}e^{-\xi_0|\xi|}$, applying the inverse Fourier transform, and changing signs.

Now, starting from the corrected form of equation 37 we have,
\begin{align}
&\phantom{=} -\beta \mathcal{F}^{-1}\left[ \frac{1}{k}e^{-i\sgn(k) k \zeta} \int_0^\zeta \mathcal{F}\left(\frac{\partial \tilde{Q}}{\partial \xi}\right) \sin k\zeta' d\zeta' + \frac{1}{k}\sin k \zeta \int_\zeta^\infty \mathcal{F}\left(\frac{\partial \tilde{Q}}{\partial \xi}\right) e^{-i\sgn(k)k\zeta'} d\zeta' \right]
\label{Eq:start} \\
&=-\frac{\beta}{2} \tilde{A} i e^{-i\tau} \int_{-\infty}^{0} e^{ik\xi} \left[\frac{1}{k}e^{i k \zeta} \int_0^\zeta  e^{\xi_0 k} e^{-\zeta} \sin k\zeta' d\zeta' + \frac{1}{k}\sin k \zeta \int_\zeta^\infty  e^{\xi_0 k} e^{-\zeta} e^{ik\zeta'} d\zeta' \right] dk \\
&\phantom{=} -\frac{\beta}{2} \tilde{A} i e^{-i\tau} \int_{0}^{\infty} e^{ik\xi} \left[\frac{1}{k}e^{-i k \zeta} \int_0^\zeta  e^{-\xi_0 k} e^{-\zeta} \sin k\zeta' d\zeta' + \frac{1}{k}\sin k \zeta \int_\zeta^\infty  e^{-\xi_0 k} e^{-\zeta} e^{-i k\zeta'} d\zeta' \right] dk \\
&= -\frac{\beta}{2} \tilde{A} i e^{-i\tau} \int_{0}^{\infty} e^{-ik\xi} \left[\frac{1}{k}e^{-i k \zeta} \int_0^\zeta  e^{-\xi_0 k} e^{-\zeta} \sin k\zeta' d\zeta' + \frac{1}{k}\sin k \zeta \int_\zeta^\infty  e^{-\xi_0 k} e^{-\zeta} e^{-ik\zeta'} d\zeta' \right] dk \\
&\phantom{=} -\frac{\beta}{2} \tilde{A} i e^{-i\tau} \int_{0}^{\infty} e^{ik\xi} \left[\frac{1}{k}e^{-i k \zeta} \int_0^\zeta  e^{-\xi_0 k} e^{-\zeta} \sin k\zeta' d\zeta' + \frac{1}{k}\sin k \zeta \int_\zeta^\infty e^{-\xi_0 k} e^{-\zeta} e^{-i k\zeta'} d\zeta' \right] dk \\
&=-\beta i e^{-i\tau} \tilde{A} \int_{0}^{\infty} \cos\left(k\xi \right) e^{-\xi_0 k} \left[\frac{1}{k}e^{-i k \zeta} \int_0^\zeta e^{-\zeta'}  \sin k\zeta' d\zeta' + \frac{1}{k}\sin k \zeta \int_\zeta^\infty e^{-\zeta'} e^{-ik\zeta'} d\zeta' \right] dk \label{Eq:rotNotes} \\ 
&=-\beta \int_{0}^{\infty} \cos\left(k\xi \right) \left[e^{-i k \zeta} \int_0^\zeta \left(\frac{\partial \tilde{Q}}{\partial \xi}\right) \sin k\zeta' d\zeta' + \sin k \zeta \int_\zeta^\infty \left(\frac{\partial \tilde{Q}}{\partial \xi}\right) e^{-ik\zeta'} d\zeta' \right] \frac{dk}{k}. \label{Eq:rotNotes2}
\end{align}
Note (\ref{Eq:rotNotes2}) matches what's in Rotunno's notes. From (\ref{Eq:rotNotes}) we have
\begin{align*}
&\phantom{=} -\beta i e^{-i\tau} \tilde{A} \int_{0}^{\infty} \cos\left(k\xi \right) e^{-\xi_0 k} \left[\frac{1}{k}e^{-i k \zeta} \int_0^\zeta e^{-\zeta'}  \sin k\zeta' d\zeta' + \frac{1}{k}\sin k \zeta \int_\zeta^\infty e^{-\zeta'} e^{-ik\zeta'} d\zeta' \right] dk \\ 
& = -\beta \tilde{A} \int_{0}^{\infty} \cos\left(k\xi \right) e^{-\xi_0 k} \left[\frac{1}{k} i e^{-i\tau} e^{-i k \zeta} \left(-\frac{e^{-\zeta}\left(\sin\left(k\zeta\right) + k\cos\left(k\zeta\right)\right)}{k^2+1} + \frac{k}{k^2+1} \right) \right. \\ &\phantom{=} \left. + i e^{-i\tau} \frac{1}{k}\sin k \zeta \left( \frac{1-ik}{k^2+1} e^{\left(-i k - 1\right)\zeta} \right) \right] dk
\end{align*}
and the real part of this is therefore
\begin{align*}
& = -\beta \tilde{A} \int_{0}^{\infty} \cos\left(k\xi \right) e^{-\xi_0 k} \frac{1}{k^2+1} \left[\sin\left(\tau +k\zeta \right)-{e}^{-\zeta}\sin\left(\tau \right) \right].
\end{align*}

My attempt at deriving equation (\ref{Eq:target}) - equation (38) in Rotunno's paper - is very messy. It didn't quite work originally as I was using the incorrect version of (37). Now having the correct version, let's see if it works. We have from (\ref{Eq:start}),
\begin{align}
&\phantom{=} -\beta \mathcal{F}^{-1}\left[ \frac{1}{k} e^{-i\sgn(k) k \zeta} \int_0^\zeta \left(\pi e^{-\xi_0 |k|} \tilde{A} e^{-\zeta} i e^{-i\tau} \right) \sin k\zeta' d\zeta' \right. \\
& \phantom{=} \left. + \frac{1}{k} \sin k \zeta \int_\zeta^\infty \left(\pi e^{-\xi_0 |k|} \tilde{A} e^{-\zeta} i e^{-i\tau} \right) e^{-i\sgn(k)k\zeta'} d\zeta' \right] \\
& = -\beta \tilde{A} \pi i e^{-i\tau} \mathcal{F}^{-1} \left[ \frac{1}{k} e^{-i\sgn(k) k \zeta} e^{-\xi_0 |k|} \int_0^\zeta e^{-\zeta'} \sin k\zeta' d\zeta' \right. \\
& \phantom{=} \left.  + \frac{1}{k} \sin k \zeta e^{-\xi_0 |k|} \int_\zeta^\infty e^{-\zeta'} e^{-i\sgn(k)k\zeta'} d\zeta' \right]  \\
& =-\beta \tilde{A} \pi i e^{-i\tau} \mathcal{F}^{-1} \left\{ \frac{1}{k} e^{-i\sgn(k) k \zeta} e^{-\xi_0 |k|}   \left[ \frac{-e^{-\zeta'}}{k^2+1}\left(k\cos k\zeta' + \sin k\zeta' \right)\right]_0^\zeta \right\} \label{Eq:int1} \\
& \phantom{=} -\beta\tilde{A} \pi i e^{-i\tau} \mathcal{F}^{-1}\left\{\frac{1}{k} \sin k \zeta e^{-\xi_0 |k|} \left[ \frac{1}{-i\sgn(k)k-1}e^{\left(-i\sgn(k)k-1\right)\zeta'} \right]_\zeta^\infty \right\}, \\
& =-\beta \tilde{A} \pi i e^{-i\tau} \mathcal{F}^{-1} \left\{ \frac{1}{k} e^{-i\sgn(k) k \zeta} e^{-\xi_0 |k|}   \left( \frac{-e^{-\zeta}}{k^2+1}\left(k\cos k\zeta + \sin k\zeta \right) + \frac{k}{k^2+1} \right) \right\} \label{Eq:term1} \\
& \phantom{=} -\beta \tilde{A} \pi i e^{-i\tau} \mathcal{F}^{-1}\left\{ \frac{1}{k} \sin k \zeta e^{-\xi_0 |k|} \left[ \frac{1}{-i\sgn(k)k-1}e^{\left(-i\sgn(k)k-1\right)\zeta'} \right]_\zeta^\infty \right\},
\end{align}
where the first integral (\ref{Eq:int1}) can be calculated by performing integration by parts. Consider the first term of the sum, i.e. (\ref{Eq:term1}). We have
\begin{align}
& \phantom{=} -\beta \tilde{A} \pi i e^{-i\tau} \mathcal{F}^{-1} \left\{ \frac{1}{k} e^{-i\sgn(k) k \zeta} e^{-\xi_0 |k|}   \left( \frac{-e^{-\zeta}}{k^2+1}\left(k\cos k\zeta + \sin k\zeta \right) + \frac{k}{k^2+1} \right) \right\} \\
&=-\beta \tilde{A} i e^{-i\tau} \frac{1}{2} \int_{-\infty}^\infty \frac{1}{k} e^{i k \xi} e^{-i\sgn(k) k \zeta} e^{-\xi_0 |k|}   \left( \frac{-e^{-\zeta}}{k^2+1}\left(k\cos k\zeta + \sin k\zeta \right) + \frac{k}{k^2+1} \right) dk \\
&=-\beta \tilde{A} i e^{-i\tau} \frac{1}{2} \int_{-\infty}^0 \frac{1}{k} e^{i k \xi} e^{i k \zeta} e^{\xi_0 k}   \left( \frac{-e^{-\zeta}}{k^2+1}\left(k\cos k\zeta + \sin k\zeta \right) + \frac{k}{k^2+1} \right) dk \\
&\phantom{=}  -\beta \tilde{A} i e^{-i\tau} \frac{1}{2} \int_0^\infty \frac{1}{k} e^{i k \xi} e^{-i k \zeta} e^{-\xi_0 k}   \left( \frac{-e^{-\zeta}}{k^2+1}\left(k\cos k\zeta + \sin k\zeta \right) + \frac{k}{k^2+1} \right) dk \\
&=-\beta \tilde{A} i e^{-i\tau} \frac{1}{2} \int_{0}^\infty \frac{1}{-k} e^{-ik \xi} e^{-ik \zeta} e^{-\xi_0k}   \left( \frac{-e^{-\zeta}}{k^2+1}\left(-k\cos k\zeta - \sin k\zeta \right) - \frac{k}{k^2+1} \right) dk \\
&\phantom{=}  -\beta \tilde{A} i e^{-i\tau} \frac{1}{2} \int_0^\infty \frac{1}{k} e^{i k \xi} e^{-i k \zeta} e^{-\xi_0 k}   \left( \frac{-e^{-\zeta}}{k^2+1}\left(k\cos k\zeta + \sin k\zeta \right) + \frac{k}{k^2+1} \right) dk \\
&= -\beta \tilde{A} i e^{-i\tau} \frac{1}{2} \int_0^\infty \frac{1}{k} \left(e^{i k \xi} + e^{-ik\xi} \right) e^{-i k \zeta} e^{-\xi_0 k}   \left( \frac{-e^{-\zeta}}{k^2+1}\left(k\cos k\zeta + \sin k\zeta \right) + \frac{k}{k^2+1} \right) dk  \\
&= -\beta \tilde{A} \left(i\cos\tau+\sin\tau\right) \int_0^\infty \frac{1}{k} \cos k\xi e^{-i k \zeta} e^{-\xi_0 k}   \frac{1}{k^2+1}\left(-e^{-\zeta}\left(k\cos k\zeta + \sin k\zeta \right) + k \right) dk \\
&= -\beta \tilde{A} \left(i\cos\tau+\sin\tau\right) \int_0^\infty \frac{1}{k} \cos k\xi \left(\cos k\zeta - i\sin k\zeta \right) \\
&\phantom{=} \times e^{-\xi_0 k} \frac{1}{k^2+1}\left(-e^{-\zeta}\left(k\cos k\zeta + \sin k\zeta \right) + k \right) dk \\
&= -\beta \tilde{A} \int_0^\infty \frac{1}{k} \sin k\xi \left(i\cos\tau+\sin\tau\right)\left(\cos k\zeta - i\sin k\zeta \right) \\
&\phantom{=} \times e^{-\xi_0 k}   \frac{1}{k^2+1}\left(-e^{-\zeta}\left(k\cos k\zeta + \sin k\zeta \right) + k \right) dk \\
&= -\beta \tilde{A} \int_0^\infty \frac{1}{k} \sin k\xi \left(i\cos\tau\cos k\zeta +  \cos\tau\sin k\zeta + \sin\tau\cos k\zeta - i\sin\tau\sin k \zeta\right) \\
&\phantom{=} \times e^{-\xi_0 k}   \frac{1}{k^2+1}\left(-e^{-\zeta}\left(k\cos k\zeta + \sin k\zeta \right) + k \right) dk
\end{align}
The real part of this is then
\begin{align}
&\phantom{=} -\beta \tilde{A}  \int_0^\infty \frac{1}{k} \sin k\xi \left(\cos\tau \cos k\zeta  + \sin\tau \sin k\zeta\right) e^{-\xi_0 k}   \frac{1}{k^2+1}\left(-e^{-\zeta}\left(k\cos k\zeta + \sin k\zeta \right) + k \right) dk \\
&= -\beta \tilde{A}  \int_0^\infty \frac{1}{k} \sin k\xi \sin\left(\tau + k\zeta\right) e^{-\xi_0 k}   \frac{1}{k^2+1}\left(-e^{-\zeta}\left(k\cos k\zeta + \sin k\zeta \right) + k \right) dk \\
& = -\beta \tilde{A}  \int_0^\infty \frac{1}{k} \sin k\xi \frac{1}{k^2+1} e^{-\xi_0 k} \\
&\phantom{=} \times \left(-\sin\left(\tau+k\zeta\right)e^{-\zeta}k\cos k\zeta - \sin\left(\tau+k\zeta\right) e^{-\zeta} \sin k\zeta + k\sin\left(\tau+k\zeta\right) \right) dk. \label{Eq:realTermOne} 
\end{align}
Consider now the second term. We have
\begin{align}
& \phantom{=} -\beta\tilde{A} \pi i e^{-i\tau} \mathcal{F}^{-1}\left\{\frac{1}{k} \sin k \zeta e^{-\xi_0 |k|} \left[ \frac{1}{-i\sgn(k)k-1}e^{\left(-i\sgn(k)k-1\right)\zeta'} \right]_\zeta^\infty \right\} \\
&= -\beta\tilde{A} i e^{-i\tau} \frac{1}{2} \int_{-\infty}^{0} \frac{1}{k} e^{i k \xi} \left(\sin k \zeta e^{\xi_0 k} \left[ \frac{1}{i k-1}e^{\left(ik-1\right)\zeta'} \right]_\zeta^\infty \right) dk \\
&\phantom{=} -\beta \tilde{A} i e^{-i\tau} \frac{1}{2} \int_{0}^{\infty} \frac{1}{k} e^{i k \xi} \left(\sin k \zeta e^{-\xi_0 k} \left[ \frac{1}{-ik-1}e^{\left(-ik-1\right)\zeta'} \right]_\zeta^\infty \right) dk \\
&= -\beta \tilde{A} i e^{-i\tau} \frac{1}{2} \int_{-\infty}^{0} \frac{1}{k} e^{i k \xi} \left(\sin k \zeta e^{\xi_0 k} \frac{1}{1-i k}e^{\left(ik-1\right)\zeta} \right) dk \\
&\phantom{=} -\beta \tilde{A} i e^{-i\tau} \frac{1}{2} \int_{0}^{\infty} \frac{1}{k} e^{i k \xi} \left(\sin k \zeta e^{-\xi_0 k} \frac{1}{1+ik}e^{\left(-ik-1\right)\zeta} \right) dk \\
&= -\beta \tilde{A} i e^{-i\tau} \frac{1}{2} \int_{0}^{\infty} \frac{1}{-k} e^{-i k \xi} \left(-\sin k \zeta e^{-\xi_0 k} \frac{1}{1+i k}e^{\left(-ik-1\right)\zeta} \right) dk \\
&\phantom{=} -\beta \tilde{A} i e^{-i\tau} \frac{1}{2} \int_{0}^{\infty} \frac{1}{k} e^{i k \xi} \left(\sin k \zeta e^{-\xi_0 k} \frac{1}{1+ik}e^{\left(-ik-1\right)\zeta} \right) dk \\
&= -\beta \tilde{A}i e^{-i\tau} \int_{0}^{\infty} \frac{1}{k} \cos k\xi e^{-\xi_0 k} e^{-ik\zeta} \left(\sin k \zeta \frac{1-ik}{1+k^2}e^{-\zeta} \right) dk \\
&= -\beta \tilde{A}i e^{-i\tau} \int_{0}^{\infty} \frac{1}{k} \cos k\xi e^{-\xi_0 k} \left(\cos k\zeta - i\sin k \zeta\right) \left(1-ik\right)\left(\sin k \zeta \frac{1}{1+k^2}e^{-\zeta} \right) dk \\
&= -\beta \tilde{A} \int_{0}^{\infty} \frac{1}{k} \cos k\xi e^{-\xi_0 k}\left(\sin k \zeta \frac{1}{1+k^2}e^{-\zeta} \right) \\
&\phantom{=}\times \left(i\cos\tau + \sin\tau \right)\left(\cos k\zeta -ik\cos k\zeta - i\sin k\zeta - k\sin k\zeta \right) dk.
\end{align}
The real part of this is then
\begin{align}
&\phantom{=} -\beta \tilde{A} \int_{0}^{\infty} \frac{1}{k} \sin k\xi e^{-\xi_0 k}\left(\sin k \zeta \frac{1}{1+k^2}e^{-\zeta} \right) \\
&\phantom{=}\times \left( k\cos\tau\cos k\zeta + \cos\tau\sin k\zeta + \sin\tau\cos k\zeta -k\sin\tau\sin k\zeta \right) dk \\
&= -\beta \tilde{A} \int_{0}^{\infty} \frac{1}{k} \sin k\xi e^{-\xi_0 k}\left(\sin k \zeta \frac{1}{1+k^2}e^{-\zeta} \right) \\
&\phantom{=}\times \left(k\cos\left(\tau + k\zeta\right) + \sin\left(\tau+k\zeta \right) \right) dk \\
&= -\beta \tilde{A} \int_{0}^{\infty} \frac{1}{k} \sin k\xi e^{-\xi_0 k}\frac{1}{1+k^2} \\
&\phantom{=} \times\left(\sin k \zeta e^{-\zeta} k \cos\left(\tau+k\zeta \right) + \sin k \zeta e^{-\zeta}\sin\left(\tau+k\zeta \right) \right) dk \label{Eq:realTermTwo}
\end{align}
Summing (\ref{Eq:realTermOne}) and (\ref{Eq:realTermTwo}) gives
\begin{align}
\tilde{\psi} (\xi,\zeta,\tau) = -\beta \tilde{A} \int_0^\infty \frac{\cos k\xi e^{-\xi_0 k }}{1+k^2} \left(\sin(k\zeta + \tau) - e^{-\zeta}\sin\tau \right) dk
\end{align}
as required!

\section{Calculating Potential Temperature}
From equation (4) \citet{rotunno83} we have 
\begin{align*}
\frac{\partial b}{\partial t} + N^2 w = Q. 
\end{align*}
The non-dimensional form of this is
\begin{align*}
& \phantom{=} \frac{\partial \tilde{b}}{\partial \tau}h \omega^3 + N^2 \tilde{w}h\omega = \tilde{Q}h\omega^3 \\
& = \frac{\partial \tilde{b}}{\partial \tau} + \left(\frac{N}{\omega}\right)^2 \tilde{w} = \tilde{Q}. 
\end{align*}
Note that $\left(\frac{N}{\omega}\right)^2$ is essentially the Berger number with $H=L$. Thus can solve for $\tilde{b}$ using
\begin{align*}
&\phantom{=} \frac{\tilde{b}_{k+1}-\tilde{b}_{k}}{\Delta \tau} = \tilde{Q}_k - \left(\frac{N}{\omega}\right)^2 \tilde{w}_k \\
&= \tilde{b}_{k+1}-\tilde{b}_{k} = \Delta \tau\left(\tilde{Q}_k - \left(\frac{N}{\omega}\right)^2 \tilde{w}_k \right).
\end{align*}
This produces a linear system of $\tau_N$ linearly independent equations in $\tau_N$ unknowns. However, in this form the system is singular - so substitute the equation for $\tilde{b_1}$ for $\tilde{b_1}+\tilde{b_\frac{\tau_n}{2}}=0$ to impose symmetry on the bouyancy. Note can use $\tilde{b_1}+\tilde{b}_{\left\lfloor\frac{\tau_n}{2}\right\rfloor}$! This works - we can solve for bouyancy even without initial conditions! From bouyancy can extract potential temperature!

Note that using this method appears to produce potential temperature perturbations that are two large, i.e. $\pm 40$ K, or even larger! Compare this with the WRF simulation of \citet{vincent16} producing perturbations of $\pm 4$ K. Could it be that the WRF data are composites, and that the signal is being diluted? Note that increasing $h$ decreases $\theta'$. Error possibly due to definition of $\bar{\theta}$ in code? 

\section{Choosing $\tilde{A}$}
Note that from equation $(4)$ we have at the surface
\begin{equation*}
\frac{\partial b}{\partial t} = \frac{g}{\theta_0}\frac{\partial \theta'}{\partial t}= Q,
\end{equation*}
as $w=0$ at the surface. Thus letting $\theta_M$ and $\theta_m$ denote the land surface temperature maxima and minima respectively, and noting that $\left(\frac{\pi}{2} + \tan^{-1} \frac{x}{x_0} \right)$ maps onto $\left(0, \pi \right)$, we have
\begin{align*}
\frac{g}{\theta_0}\frac{\theta_M - \theta_m}{12\cdot 60\cdot 60} =  \left( \sin\frac{\pi}{2} - \sin\frac{3\pi}{2} \right)A\pi = 2 A \pi,
\end{align*}
with $A \pi$ the supremum of $H$ at the surface, and it taking 12 hours to go from maximum to minimum temperature. Multiplying both sides by $h^{-1} \omega^{-3}$ gives  
\begin{align*}
\tilde{A} = \frac{g}{2 \pi \theta_0}\frac{\theta_M - \theta_m}{12\cdot 60\cdot 60} h^{-1} \omega ^{-3}. 
\end{align*}

\bibliography{../../Ewan_Short_Masters/Bibliography/ewansbibli.bib}

\end{document}
